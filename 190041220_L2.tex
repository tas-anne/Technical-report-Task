\documentclass{article}
\usepackage[margin = 1in]{geometry}
\usepackage{amsmath}
\usepackage{amssymb}
\title{\textbf{Differential Operator}}
\author{Tasfia Tasneem Annesha 190041220}
\date{January 2023}

\begin{document}

\maketitle
The operator representing the computation of a derivative,

\\
\begin{align}
\tilde{D} \equiv \frac{d}{dx} ,
\end{align}

sometimes also called the Newton-Leibniz operator. The second derivative is then denoted $\tilde{D}^2$ , the third $\tilde{D} ^3$, etc. The integral is denoted $\tilde{D}^{-1}$.
\\
\\
The differential operator satisfies the identity
\begin{align}
     \left(2x- \frac{d}{dx} \right)^n 1= H_n\left(x\right),
\end{align}
where \textit{$H_n(x)$} is a Hermite polynomial (Arfken 1985, p. 718), where the first few cases are given explicitly by
\begin{align}
    H_1(x)&=2x- \frac{\delta 1}{\delta x}\\
          & =2x\\
    H_2(x)&=2x(2x)- \frac{\delta (2x)}{\delta x}\\
          & =4x^2 -2\\
    H_3(x)&=2x(4 x^2 - 2)- \frac{\delta (4x^2 -2)}{\delta x}\\
          & =8x^3 - 12x.      
\end{align}
The symbol $\vartheta$ can be used to denote the operator
\\
\begin{align}
    \vartheta \equiv x \frac{d}{dx}
\end{align}
(Bailey 1935, p. 8). A fundamental identity for this operator is given by\\
\begin{align}
    \left(x\Tilde{D}\right)^n = \sum_{k=0}^{n}S(n,k)\hspace{0.05cm}x^k \Tilde{D}^k ,
\end{align}
\\
where \textit{S(n,k)} is a Stirling number of the second kind (Roman 1984, p. 144), giving
\begin{align}
    \left(x\Tilde{D}\right)^1 &= x\Tilde{D} \\
    \left(x\Tilde{D}\right)^2 &= x\Tilde{D} + x^2\Tilde{D} ^2\\ 
    \left(x\Tilde{D}\right)^3 &= x\Tilde{D} + 3x^2\Tilde{D} ^2 + x^3\Tilde{D} ^3\\ 
    \left(x\Tilde{D}\right)^4 &= x\Tilde{D} + 7x^2\Tilde{D} ^2 + 6x^3\Tilde{D} ^3 + x^4\Tilde{D}^4
\end{align}
and so on (OEIS A008277). Special cases include
\begin{align}
    \vartheta^n e^x &=e^x \sum_{k=0}^{n}S(n,k)x^k\\
    \vartheta^n \cos x &=\cos x \sum_{k=0}^{n}(-1)^k \hspace{0.1cm}S(n, 2k) x^{2k} + \sin x \sum_{k=1}^{n}(-1)^k \hspace{0.1cm} S(n,2k-1) x^{2k-1}\\
    \vartheta^n \sin x &=\cos x \sum_{k=1}^{n}(-1)^{k+1} \hspace{0.1cm}S(n, 2k - 1) x^{2k- 1} + \sin x \sum_{k=0}^{n}(-1)^k \hspace{0.1cm} S(n,2k) x^{2k}.
\end{align}
A shifted version of the identity is given by
\begin{align}
    [ \,(x-a) \Tilde{D} ] \,^n = \sum_{k=0}^{n} S(n,k)(x-a)^k \hspace{0.1cm}\Tilde{D}^k
\end{align}
(Roman 1984, p. 146).
 \end{document}

